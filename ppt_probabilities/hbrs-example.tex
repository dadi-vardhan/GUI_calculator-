\documentclass{beamer}


\usepackage[utf8]{inputenc}
\usepackage{amsmath}
\usepackage{amsfonts}
\usepackage{amssymb}
\usepackage{graphicx}
\usepackage{ragged2e}  % `\justifying` text
\usepackage{booktabs}  % Tables
\usepackage{tabularx}
\usepackage{tikz}      % Diagrams
\usetikzlibrary{calc, shapes, backgrounds}
\usepackage{amsmath}
\usepackage{amssymb}
\usepackage{dsfont}
\usepackage{url}       % `\url
\usepackage{listings}  % Code listings
\usepackage[T1]{fontenc}


\usepackage{theme/beamerthemehbrs}

\author[Name]{ Krishna Teja\\  Vishnu Vardhan\\  Manoj Kumar \\ Hammam Abdelwahab}
\title{Software Development Project}
\subtitle{Uncertain<T> : Porting library from C\# to Python and Cpp}
\institute[HBRS]{Hochschule Bonn-Rhein-Sieg}
\date{\today}
\subject{Test beamer}

% leave the value of this argument empty if the advisors
% should not be included on the title slide
\def\advisors{Deebul Nair}

% \thirdpartylogo{path/to/your/image}


\begin{document}
{
\begin{frame}
\titlepage 
\end{frame}
}

\begin{frame}{Sampling}
\begin{itemize}
\item Sampling is the process of generating values of a random variable  $X$  by taking the probability distribution of  $X$  into account.
\item We can be certain that we are generating samples from a given distribution if, as  $x \to \infty$, the distribution of the samples resembles the original distribution.

\item More explanation will show with this demo. 
\end{itemize}

\end{frame}

\begin{frame}

\end{frame}




\end{document}
